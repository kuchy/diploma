{\noindent\large\bf Abstrakt}

\vspace{1.8cm}
V diplomovej práci sú navrhnuté a zvalidované metódy pre spracovanie videa a detekciu významných oblastí vo videu.
Navrhovaná metóda zameriava na získanie čisto dynamických príznakov, ktoré je možné následne kombinovať s klasickými príznakmi alebo použiť ako samostaný príznak pri vytváraní významných oblastí obrazu.
Pre uľahčenie prototypovania podobných modelov, bola vytvorená ucelená aplikácia v prostredí Matlab, poskytujúca automatickú validáciu modelu pomocou štandardných datasetov a jednoduché vkladanie konkurenčných modelov pre okažité porovnávanie.
Práca je rozdelená do piatich kapitol a zaoberá sa výskumom alternatívnych spôsobov detekcie významných oblastí pre potreby ďalšieho spracovania obrazu.
\\

{\parindent0pt \textbf{Kľúčové slová:} \emph{významné oblasti, video, Matlab}}

\newpage
 {\noindent\large\bf Abstract}
  \vspace{1.8cm}


In the diploma thesis, we propose and validate methods for processing video and detection of salient areas in it. Suggested method are focuses on acquirement of purely dynamic attributes, which can be subsequently combined with classic attributes or can be used as separate feature for generating salient maps.
To simplify prototyping of similar models, a comprehensive application was created in Matlab environment, which is providing automatic validation of model using standard datasets and simple input of competitive models for immediate comparison.
In five chapters, we deal with research of alternative approaches of detection of salient areas for requisites of further image processing.

{\parindent0pt \textbf{Keywords:} \emph{saliency, video, Matlab}}


\newpage	
