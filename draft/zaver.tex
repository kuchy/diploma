Cieľom diplomovej práce bolo preskúmať možnoti extrakcie príznakov významných oblastí pre videá, pomocou príznakou ktoré nemožno extrahovať z čisto obrazovej informácie a navrhnúť metódu používajúcu takéto príznaky.
Ďalším cieľom bolo vypracovať validáciu navrhovaného modelu pomocou štandartných metrík používaných pri hodnotení modelov na detekciu významných oblastí.
\\
\\
V práci som navrhol a adekvátne zvalidoval metódu, ktorá využíva kombináciu štandartných metód pre detekciu významných oblastí vo videu.
Naviac bol vypracovaný benchmark, ktorý obsahuje šešť iných modelov/algoritmov za účelom porovnania a analýzy výsledkov navrhovaného modelu.
Sekundárnym prínosom práce je ucelená applikácia, ktorá má potenciál výrazne zjednodušiť prototypovanie, testovanie a validovanie podobných modelov pre budúcich študentov.
Aplikácia je navrhnutá pre výrazné ušetrenie práce, vďaka jednoduchému pridávaniu modelov pozornosti a možnosti ich vizuálneho porovnania.
Taktiež je jej pomocou možné validovať výsledky pomocou oficiálne publikovaných troch datasetov a troch štandardne používaných metrík.
Validácia prebieha automaticky a dokáže z výsledkov aj generovať výsledky vo forme grafov používaných aj v tejto práci.
Aplikácia je dostupná v elektronickej prílohe a voľne dostupná na internete. 
\\
\\
Ďalší možný rozvoj je popísaný a analyzovaný v sekcii \ref{ssec:diskusia}.
