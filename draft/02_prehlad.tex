\\\\
\section{Úvod do problematiky}
Saliency a teda detekcia významných oblatí je využívaná roznych oblastiach. Počínajúc automatizáciou je tažiskom pri segmentácií obrazu alebo detekcíi špecifických objektov. Od saliency modelov sú taktiež závyslé aj programy ovládajúce zabezpečovacie zariadenia. Až po reklamu kde je vyzuálna pozornosť klúčovým parametrom čo može rozhodnúť o úspechu produktu, veď aký význam by mala reklama kde si nevšimnete prezentovaný produkt.
\section{Metody pre statické obrázky}
Algoritmy pre statické oblasti tvoria základ všetkých saliency modelov a tvoria najstaršiu oblasť výskumu. V tejto časti uvediem prehlad algoritmov pre výpočet saliency modelov od najjednoduchších cez najznámejšie až po nejefektívnejšie.
\subsection{Baseline Center}
\subsection{Hrany}
\subsection{Prechody}
\subsection{Ittiho model}
\subsection{spektralne rezidua}
\subsection{Judd Model}
\subsection{Rare Model}
...Context-Aware saliency, Weighted Maximum Phase Alignment Model, Torralba saliency, Murray model,
\section{Metody pre videá}
Video obsahuje rozsiahlejšie možnosti ako iba obrazová informácia, pribúdajú dalšie rozmery ako je pohyb objektov na obraze alebo vplyv zvuku na ľudské vnímanie. Avšak oproti obrazu obsahuje je potrebné spracovávať vedšie množtvo
dát. Navyše vo vedšine algoritmov využívajúcich saliency modely je potrebné aby model dával výsledky v reálnom čase. Používané hlavne v oblasti zabezpečovacej techniky.
\subsection{Zohladnenie audio informácie}
1)http://ieeexplore.ieee.org/stamp/stamp.jsp?tp=&arnumber=6616164
2)http://www.gipsa-lab.grenoble-inp.fr/~antoine.coutrot/Coutrot_ICIP2014.pdf
3)http://www.gipsa-lab.grenoble-inp.fr/~antoine.coutrot/Coutrot_2013_Annals_of_Telecommunications.pdf
\subsection{Detekcia pohybu}
V tejto časti sa zmeriame na segmentáciu objektov ktoré sa na scéne pohybujú. Taktéto obrazy sú v ľudskom vyzuálnom systéme vysoko hodnotené. Dôvody, prečo takto ludský vyzuálny systém pridáva prijoritu práve takýmto oblastiam možeme nájsť v antropológií (najsť zdroj!). Vysvetlenie je jednoduché a to snaha zabezpečit bezpečné prostredie okolo seba a všetko pohybujúce sa narušuje pocit bezpečnosti. V nasledujúcom texte rozoberieme 2 najpožívanejšie algoritmy používané na detekciu oblastí pohybu v obraze. Prvým z nich bude LUCAS KANADE[anotacia!], a druhým Horn Schunck[anotacia!].
TODO:
#http://ieeexplore.ieee.org/xpl/login.jsp?tp=&arnumber=4269999&url=http%3A%2F%2Fieeexplore.ieee.org%2Fxpls%2Fabs_all.jsp%3Farnumber%3D4269999
\section{Metódy Využívajúce neurónové siete}
\section{Refenčné datasety}
RSD, SAVAM, AUDITORY DATASET
\section{Metriky úspešnosti}
Zobrat z MIT saliency
\section{Porovnanie štandardných Metód}
