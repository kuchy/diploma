\section{Platforma pre riešenie}
Ako platfomu pre implementáciu bude použitý programovací jazyk Matlab\textregistered.
Všetky zdrojové kódy budú poskytnuté výhradne v tomto vývojárskom jazyku.
\section{Očakávané výsledky}
Výsledkom práce bude model pozornosti, ktorý zohľadňuje príznaky extrahovateľné iba z videa a nie z čisto obrazovej informácie.
Pôjde o pohyb objektov na scéne a iné sémantické informácie, ktorými sa video odlišuje od statickej scény.
Sekundárnym prínosom práce bude vytvorenie jednotnej aplikácie pre vizuálne porovnávanie modelov, kde používateľ bude môcť jednoducho pridávať modely. Ideálne priamo pouźiť ukážkové zdrojové kódy zverejnené autormi jednotlivých modelov alebo úpravou, ktorá nevyžaduje znalosť logiky stojacej za daným modelom.
Následne automatický výpočet štadardných metrík na implementovanom datasete, pre jednoduchú validáciu výsledkov na rovnakých dátach, spolu s konkurečnými modelmi z dôvodu jednoduchého ladenia počas vývoja modelu.
\section{Ideálne prípady}
Ideálne prípady sa očakávajú v prípade použitia záznamov z bezpečnostných kamier, z dôvodu statickej kamery.
Vďaka statickému pozadiu sú výsledky detekcie optického toku objektov najrelevantnejšie, čo predurčuje takéto videá k najlepším výsledkom.
\section{Problémové prípady}
Očakávame, že najproblémovejšími vstupmi budú videá s dynamickým pohybom kamery kombinovaným s pohybom objektov, alebo videá s fixovanou kamerou na objekt dynamicky sa pohybúci po scéne.
Vo videách takéhoto charakteru sa predpokladá chybné označovanie oblastí a z toho vyplývajúce nepresnosti v mapách pozornosti.
Preto v týchto prípadoch budú utlmované dynamické príznaky videa a budú sa používať iba statické príznaky aktuálneho obrazu.
