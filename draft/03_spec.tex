\section{Platforma pre riešenie}
Ako platfoma pre implementáciu budeme používať programovací jazyk Matlab(uviest citaciu?).
\section{Očakávané výsledky}
Výsledkom práce bude model pozornosti ktorý, zohladnuje príznaky extrahovateľné iba z videa a nie z čisto obrazovej inforácie.
Pôjde hlavne o pohyb objektov na scéne a iné sémantické informácie ktorímy sa video odlišuje od statickej scény.
Príkladom nového rozmenu videa okrem možnosti pohybu objektov je napríklad aj pameť užívateľa s ktorou musíme počítať pri tvorbe mapy pozornosti.
Ide o jednoduchý princíp a to ten, že objekty nachádzajúce sa na scéne "príliš dlho" scéne strácajú výnimočnosť a z toho vyplýva aj rozdielnosť  oproti štandardným modelom pozornosti.
Sekundárnym prínosom práce bude vytvorenie jednotnej applikácie pre vizuálne porovnávanie modelov, kde používateľ bude môcť jednoducho pridávať modeli, ideálne priamo pouźiť ukážkové zdrojové kódy zverejnené autormy jednotlivých modelov alebo úpravou ktorá nevyžduje znalosť logiky stojacej za daným modelom.
Následne automatický výpočet štadardných metrík na implementovanom datasete pre jednoduchú validáciu výsledkov na rovnakých dátach spolu s konkurečnýmy modelmy z dôvodu jednoduchého ladenia počas vývoja modelu.
\section{Ideálne Prípady}
Idealny priklad modelu (obrazok)
\section{Problémové Prípady}
Predpokladané problematické úseky
