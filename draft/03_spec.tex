\section{Platforma pre riešenie}
Ako platfoma pre implementáciu budeme používať programovací jazyk Matlab.
\section{Očakávané výsledky}
Výsledkom práce bude model pozornosti ktorý, zohladnuje príznaky extrahovateľné iba z videa a nie z čisto obrazovej inforácie.
Pôjde hlavne o pohyb objektov na scéne a iné sémantické informácie ktorímy sa video odlišuje od statickej scény.
Príkladom nového rozmenu videa okrem možnosti pohybu objektov ktorú musíme počítať pri tvorbe mapy pozornosti.
Sekundárnym prínosom práce bude vytvorenie jednotnej applikácie pre vizuálne porovnávanie modelov, kde používateľ bude môcť jednoducho pridávať modeli, ideálne priamo pouźiť ukážkové zdrojové kódy zverejnené autormy jednotlivých modelov alebo úpravou ktorá nevyžduje znalosť logiky stojacej za daným modelom.
Následne automatický výpočet štadardných metrík na implementovanom datasete pre jednoduchú validáciu výsledkov na rovnakých dátach spolu s konkurečnýmy modelmy z dôvodu jednoduchého ladenia počas vývoja modelu.
\section{Ideálne Prípady}
Idálne prípady očakávam v prípade použitia záznamov z bezpečnostných kamier, z dôvodu statickéj kamery.
Vďaka statickému pozadiu sú výsledky detekcie optického toku objektov najrelevantnejšie.
To predurčuje takého videá k najlepším výsledkom.
\section{Problémové Prípady}
Najproblémovejšími vstupmy očakávam videá s dynamickým pohybom kamery kombinovaným s pohybom objektov.
Vo videách takéhoto charakteru predpokldám chybné označovanie oblastí a z toho vyplývajúce chyby v mapách pozornosti, preto sa budem snažiť v týchto prípadoch utlmovať dynamické príznaky videa.
