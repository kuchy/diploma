Lorem ipsum dolor sit amet, consectetur adipiscing elit a další text... 
\\\\
V nasledujúcich častiach...

\section{Podkapitola o zdrojoch}
Text s uvedenym zdrojom \cite{vis}:
Zdroje sa ako bibtex(ten sa da vacsinou rovno stiahnut) pridavaju do textoveho suboru \emph{bibliography.bib} (ktory treba nasledne tiez prekompilovat) a potom sa mozu citovat. Zdroj moze byt napr typu article \cite{taxonomia},  online link a ine. Pri online linku \cite{anemone}
treba uviest datum citovania, ten sa nastavuje v subore \emph{fmph.bbx} a pri online zdrojoch sa potom uz automaticky vypise. 
\subsection{Este mensia podsekcia - zoznamy}

Zoznam poloziek:
\begin{itemize}
	\item \textbf{prve} – popis
	\item \textbf{dalsie} – Aliquam erat volutpat. Proin vel mi aliquet, dignissim metus quis, elementum lectus. Quisque ac est enim. Interdum et malesuada fames ac ante ipsum primis in faucibus.
	\item \textbf{este nieco dalsie} - dalsi popis
\end{itemize}

  Číslovaný zoznam:
  \begin{enumerate}
    \item\textbf{Analýza dát}
    \item\textbf{Niečo dalšie}
    \item\textbf{Ešte niečo}
    \item\textbf{Aby toho bolo viac} 
  \end{enumerate}
  
\subsection{Este jedna - obrazky vedla seba}
 %parameter H znamena ze obrazok ma byt vzhladom na ostatny obsah presne na tom mieste ako v zdrojovom subore,  parameter ht je volnejsi a obcas dava text okolo obrazkov resp obrazok do zvysneho obsahy podla vlastneho uvazenia...

\begin{figure}[H] 
\begin{center}
\includegraphics[scale=0.7]{pics/ilu.png} \hspace{1.5cm}
\includegraphics[scale=0.57]{pics/ilu.png}
\caption{Dva obrazky so spolocnym popisom a vlozenou horizontalnou medzerou medzi obr}
\label{fig:ch3}
\end{center}
\end{figure}

\begin{figure}[H]   
\begin{center}
\captionsetup{font=small}
\includegraphics[scale=1.0]{pics/ilu.png}
\caption{Popis obrazku s mensim fontom popisu ked mam priiilis dlhyy popis velky ako odsek a uz by sa to bilo so zvysnim textom tak nech to odlisim ze je to popis}
\label{fig:ch1}
\end{center}
\end{figure}

vlozeny vacsi vertikalny priestor medzi obrazkami
\vspace{1.6cm}  

\begin{figure}[ht] 
\begin{center}
\includegraphics[scale=1.0]{pics/ilu.png}
\caption{Dalsi obrazok s velkostou fontu popisu ako ostatny text}
\label{fig:ch2}
\end{center}
\end{figure}


\section{Ina podkapitola}
a dalsi mudry text
\subsection{A  v nej subsekcia}
