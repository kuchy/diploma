
Ľudské oko je schopné spracovať \begin{math}10^8\end{math} až \begin{math}10^9\end{math} bitov obrazových dát za sekundu.
Ľudský mozog nie je schopný spracovať také množstvo dát naraz, preto sa získané informácie filtrujú pomocou ľudského vizuálneho systému\cite{Fmph-videnie}.
Ľudský vizuálny systém je pravdepodobne najzložitejším mechanizmom akým človek disponuje.
Je natoľko kľúčovým pre fungovanie spoločnosti či jedinca, že psychológovia sa zaoberajú jeho výskumom.
Už viacero dekád študujú vlastnosti tohoto mechanizmu z pohľadu psychológie, fyziológie alebo neurobiológie.
\\
Vyfiltrované oblasti obrazu je možné spracovať vo výrazne rýchlejšom čase ako nefiltrované, ideálne v reálnom čase.
Takéto oblasti sa nazývajú významné alebo charakteristické (v~literatúre salient - prebraté z angličtiny).
Významné oblasti sú vyberané pomocou mnohých faktorov.
Najznámejšími sú prechody vo farbe, intenzite alebo orientácii.
Ľudský vizuálny systém taktiež využíva skúsenosti pri pozorovaní.
Oblasti takto vyfiltrované nesú pre pozorovateľa viac potencionálnych informácií ako ostatné oblasti obrazu a preto sa stávajú salientnými.
\\
V systémoch počítačového videnia sa snažíme využívať primárne tieto oblasti pre pridelenie väčšej časti zdrojov.
Z tohto dôvodu je zistenie významných oblastí často prvým krokom mnohých algoritmov v oblasti počítačového videnia.
\\
Algoritmy na detekciu významných oblastí sa delia do troch skupín podľa princípu akým spracovávajú dáta\cite{brief-survey}

  \begin{enumerate}
          \item Zdola-nahor: Prístup je cielený na nezávislosť od používateľa.
          Zameriava sa na~fyziologicky významné oblasti vizuálneho systému ako výrazné zmeny v tvare, jase alebo farbe.
          \item Zhora-nadol: Prístup je založený na čiastočnom riadení zo strany používateľa (konanie je podmienené úlohou).
          Riadenie je prínosom, pretože obsahuje aj informáciu používateľa a jeho predchádzajúcich vedomostí či skúseností, ktoré ovplyvňujú vnímanie.
          \item Algoritmy využívajúce neurónové siete.
  \end{enumerate}

Cieľom práce je štúdium a výskum nových metód na detekciu významných oblastí vo videu a ich porovnanie s existujúcimi v rôznych štandardných oblastiach.
\\
\textbf{V prvej časti} sa nachádza prehľad metód na detekciu významných oblastí vo videu alebo metód na detekciu v statických obrazoch, ktoré majú potenciál pre použitie aj vo videu.
Ďalej obsahuje detailné vysvetlenie fungovania metód, ktoré budú použité v implementácii zlepšenia.
\\
\textbf{V druhej časti} je popísaný postup, princíp zlepšenia a implementácie novej metódy.
\\
\textbf{Tretiu časť} tvorí validácia výsledkov a porovnanie výsledkov s inými modelmi, ktoré poskytujú lepšiu predstavu o efektivite algoritmu. A následne aj diskusiu obsahujúcu analýzu výsledkov validácie a z nej vyplývajúce možnosti na zlepšenie narvhnutého algoritmu.
\\
\textbf{Záver} obsahuje zhrnutie vytýčených cieľov práce a ich objektívne zhodnotenie.