
Ľudské oko je schopné spracovať \begin{math}10^8\end{math} až \begin{math}10^9\end{math} bitov obrazových dát za sekundu.
Ludský mozog nie je schopný spracovať také možstov dát naraz, preto sa získané informácie filtrujú pomocou ľudského vizuálneho systému\cite{Fmph-videnie}.
Ľudský vizuálny systém je pravdepodobne najzložitejším mechanizom akým človek disponuje.
Je natoľko klúčovým pre fungovanie spoločnosti či jedinca, že psychológovia sa zaoberajú jeho výskumom.
Už viacero dekád študujú vlastnosti tohoto mechanizmu z pohladu psychológie, fyziológie alebo neurobiológie.
\\
Vyfiltrované oblasti obrazu už je možne spracovať v výrazne rýchlejšie ako nefiltrované, ideálne v reálnom čase.
Takéto oblasti sa nazívajú význammné alebo charakteristické (v literatúre salient - prebrané v angličtiny).
Významné oblasti su vyberané pomocou mnoha faktorov.
Najznámejšímy su prechody vo farbe, intenzite alebo orientácií.
Ľudský vizuálný systém taktiež využíva skúsenosti pri pozorovaní.
Oblasti takto vyfiltrované nesú pre pozorovateľa viac potencionálnych informácií ako ostatné oblasti obrazu a preto sa stávajú salietnýmy.
\\
V systémoch počítačového videnia sa snažíme využívať primárne tieto oblasti pre pridelenie väčšej časti zdrojov.
Z tohoto dôvodu je zistenie zaujímavých oblasti častým prvým krokom mnohých algoritmov v oblasti počítačového videnia.
\\
Algoritmy na detekciu význammných oblastí sa delia do 3 skupín podľa princípu akým spracovávajú dáta\cite{brief-survey}

  \begin{enumerate}
          \item Zdola-nahor: Prístup je cielený na nezávyslosť od používateľa.
          Zameriava sa fyziologicky významné oblasti vizuálneho systému ako výrazné zmeny v tvare, jase alebo farbe.
          \item Zhora-nadol: Prístup je založený na čiastočnom riadení zo strany používateľa (konanie je podmienené úlohou).
          Riadenie je prínosom pretože obsajuhe aj informáciu používatela a jeho prechádzajúcich vedomostí či skúsenosti, ktoré ovplyvnujú vnímanie.
          \item Algoritmy využívajúce neurónové siete.
  \end{enumerate}

Cieľom práce je štúdium a výskum nových metód na detekciu významných oblastí vo videu.
Následne porovnanie nových métod s existujúcimi v rôznych štandardnych oblastiach ako aj v rýchlosti výpočtu.
\\
V prvej časti sa nachádza prehľad metód na detekciu významných oblastí vo videu, alebo metód na detekciu v statických obrazoch ktoré majú potenciál pre použitie aj vo videu.
Ďalej detailné vysvetlenie fungovania metód ktoré budú použité v implementácií zlepšenia.
\\
V druhej časti je popísaný postup a princíp zlepšenia a postup implementácie.
Tretiu časť tvorí validácia výsledkov a porovnanie výsledkov s inýmy modelmy, ktoré posktujú lepšiu predstavu o efektivite algoritmu.
\\
V závere....
