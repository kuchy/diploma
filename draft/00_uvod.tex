
Ľudské oko je schopné spracovať \begin{math}10^8\end{math} až \begin{math}10^9\end{math} bitov obrazových
dát za sekundu. Ludský mozog nie je schopný spracovať také možstov dát naraz,
preto sa získané informácie filtrujú pomocou ľudského vizuálneho systému. Ľudský vizuálny systém
je pravdepodobne najzložitejším mechanizom akým človek disponuje. Je natoľko klúčovým pre fungovanie spoločnosti či jedinca, že
psychológovia sa zaoberajú jeho výskumom. Už viacero dekád študujú vlastnosti tohoto
mechanizmu z pohladu psychológie, fyziológie alebo neurobiológie. \\

Vyfiltrované oblasti obrazu už je možne spracovať v reálnom čase. Takéto oblasti sa nazívajú
význammné (salient). Významné oblasti su vyberané pomocou mnoha faktorov. Najznámejšímy su prechody
vo farbe, intenzite alebo orientácií. Ľudský vizuálný systém taktiež využíva
skúsenosti pri pozorovaní. Oblasti takto vyfiltrované nesú pre pozorovateľa
viac potencionálnych informácií ako ostatné oblasti obrazu. \\

V systémoch počítačového videnia sa snažíme využivať primárne tieto oblasti pre
pridelenie väčšej časti zdrojov. Z toho dôvodu je zistenie zaujímavých oblasti
prvým krokom mnohých algoritmov v oblasti počítačového videnia.\\

Algoritmy na detekciu význammných oblastí sa delia do 3 skupín podľa princípu
akým spracovávajú dáta[Saliency in images and video: a brief survey]

  \begin{enumerate}

          \item Zdola-nahor: Prístup je cielený na nezávyslosť od používateľa.
          Zameriava sa fyziologicky významné oblasti vizuálneho systému ako
          výrazné zmeny v tvare, jase alebo farbe.

          \item Zhora-nadol: Prístup je založený na čiastočnom riadení zo
          strany používateľa (konanie je podmienené úlohou). Riadenie je prínosom pretože obsajuhe aj informáciu
          používatela a jeho prechádzajúcich vedomostí či skúsenosti, ktoré ovplyvnujú vnímanie.

          \item Algoritmy využívajúce neurónové siete.

  \end{list_type}
 \\
Cieľom práce je štúdium a výskum nových metód na detekciu významných oblastí vo videu.
Následne porovnanie nových métod s existujúcimi v rôznych štandardnych oblastiach ako aj v rýchlosti výpočtu. \\

V prvej časti sa nachádza prehľad metód na detekciu významných oblastí vo videu,
alebo metód na detekciu v statických obrazoch ktoré je možné aplikovať aj vo videu.
Ďalej detailné vysvetlenie fungovania metód ktoré budú použité v implementácií zlepšenia. \\

V druhej časti je popísaný postup a princíp zlepšenia. Následne porovnanie s metódamy
uvedenýmy v prvej časti.

\\
Ďalej.....  \\
V závere....
